\documentclass{article}

\usepackage{geometry}
  \geometry{
    a4paper,
    total={170mm,257mm},
    left=30mm,
    right=30mm,
    top=20mm,
}

\usepackage[utf8]{inputenc} % allow utf-8 input
\usepackage[T1]{fontenc}    % use 8-bit T1 fonts
\usepackage{hyperref}       % hyperlinks
\usepackage{url}            % simple URL typesetting
\usepackage{booktabs}       % professional-quality tables
\usepackage{amsfonts}       % blackboard math symbols
\usepackage{nicefrac}       % compact symbols for 1/2, etc.
\usepackage{microtype}      % microtypography

\usepackage[labelfont=bf]{caption}
\usepackage[subrefformat=parens]{subcaption}
\usepackage[usenames,dvipsnames,svgnames]{xcolor}
\usepackage{algorithmicx}
\usepackage{algorithm}
\usepackage[noend]{algpseudocode}
\usepackage{amsfonts}
\usepackage{amsmath}
\usepackage{amssymb}
\usepackage{bm}
\usepackage{chngcntr}
\usepackage{mdframed}
\usepackage{graphicx}
\usepackage{parskip}
\usepackage{placeins}
\usepackage{soul}
\usepackage{enumerate}

% fonts
\renewcommand{\rmdefault}{ptm}
\renewcommand{\sfdefault}{phv}

\newcommand*\Let[2]{\State {#1} $\gets$ {#2}}
\newcommand*\Sample[2]{\State {#1} $\sim$ {#2}}
\algnewcommand{\LineComment}[1]{\State \(\triangleright\) #1}

\newcommand{\balpha}{\bm\bm{\alpha}}
\newcommand{\bbeta}{\bm\beta}
\newcommand{\bgamma}{\bm\gamma}
\newcommand{\btheta}{\bm\bm{\Theta}}
\newcommand{\bTheta}{\bm\Theta}
\newcommand{\blambda}{\bm\lambda}
\newcommand{\G}{\mathcal{G}}
\newcommand{\R}{\mathbb{R}}
\newcommand{\T}{T}
\newcommand{\s}{\bm{s}}
\newcommand{\x}{\bm{x}}
\newcommand{\X}{\bm{X}}
\newcommand{\y}{\bm{y}}
\newcommand{\Y}{\bm{Y}}
\newcommand{\z}{\bm{z}}
\newcommand{\Z}{\bm{Z}}

\newcommand{\gpm}{\mathcal{G}}
\newcommand{\point}{\mathbf{x}}
\newcommand{\memberindex}{r}
\newcommand{\member}{\mathbf{x}_\memberindex}
\newcommand{\memberj}{x_{[\memberindex, j]}}
\newcommand{\numstates}{N_s}
\newcommand{\numtransitions}{N_t}
\newcommand{\data}{\mathcal{D}}
\newcommand{\datastar}{\data^\star}
\newcommand{\columns}{J}
\newcommand{\givens}{\mathcal{D}_g}
\newcommand{\comparison}{\mathcal{D}_c}
\newcommand{\comparisonindexes}{\mathcal{I}}
\newcommand{\comparisonmember}{\mathbf{x}_i}
\newcommand{\comparisonset}{\set{\x_{[r_i,Q]}}_{i \in \comparisonindexes}}
\newcommand{\score}{\textsc{score}}
\newcommand{\logscore}{\textsc{log score}}
\newcommand{\logpquery}{Q=\set{x_{[r,q_j]}}}
\newcommand{\values}{x_{[i,j]}}
\newcommand{\givenrow}{E=\set{x_[r,e_j]}}

\newcommand{\pop}{\mathcal{P}}
\newcommand{\mV}{\mathcal{V}}
\newcommand{\Vt}{\mathcal{V}_t}
\newcommand{\bx}{\bm{x}}
\newcommand{\bX}{\bm{X}}
\newcommand{\mD}{\mathcal{D}}
\newcommand{\mG}{\mathcal{G}}
\newcommand{\B}{\mathcal{B}}
\newcommand{\bs}{\bm{s}}
\newcommand{\by}{\bm{y}}
\newcommand{\mI}{\mathcal{I}}
\newcommand{\mO}{\mathcal{O}}
\newcommand{\out}{\mathit{out}}
\newcommand{\bZ}{\bm{Z}}
\newcommand{\bz}{\bm{z}}
\newcommand{\bY}{\bm{Y}}
\newcommand{\naturals}{\mathbb{N}}
\newcommand{\bomega}{\bm{\omega}}
\newcommand{\bphi}{\bm{\phi}}
\newcommand{\pg}{p_{\gpm}}
\newcommand\numberthis{\addtocounter{equation}{1}\tag{\theequation}}
\newcommand{\bxr}{\bx_r}
\newcommand{\bxstar}{\bx^\star}
\newcommand{\Dstar}{\text{D}^\star_n}
\newcommand{\schema}{\mathcal{S}}

\newcommand{\variable}[1]{X_{(#1)}}
\newcommand{\measurement}[1]{x_{(#1)}}
\newcommand{\latent}[1]{z_{(#1)}}
\newcommand{\query}[1]{q_{(#1)}}
\newcommand{\bxrn}[1]{\bx_{r_#1}^\star}
\newcommand{\set}[1]{\{{#1}\}}
\newcommand{\bcaption}[2]{\caption{\textbf{#1} {#2}}}

\DeclareMathOperator{\argmin}{arg\,min}
\DeclareMathOperator{\argmax}{arg\,max}
\DeclareMathOperator{\logsumexp}{logsumexp}

\def\indep{\perp\!\!\!\perp}

\newcommand{\red}[1]{\textcolor{red}{#1}}
\newcommand{\blue}[1]{\textcolor{blue}{#1}}

\frenchspacing
\setlength{\floatsep}{5pt plus 1.0pt minus 2.0pt}
\setlength{\textfloatsep}{10pt plus 1.0pt minus 2.0pt}
\setlength{\abovecaptionskip}{2pt plus 0.0pt minus 1.0pt}
% \belowcaptionskip.
\setlength{\abovedisplayskip}{5pt plus 1.0pt minus 2.0pt}
\setlength{\belowdisplayskip}{5pt plus 1.0pt minus 2.0pt}


\begin{document}
\title{Data Search for the DP Mixture}
\author{Leo Casarsa}
\maketitle

\section{Pseudocode for CrossCat Logpdf}
-- Inline datatype specific math

\begin{algorithm}[h]
  \renewcommand{\thealgorithm}{}
  \caption{\texttt{logpdf} for CrossCat}
  \begin{algorithmic}[1]
    \Function{LogPdf}{
      \texttt{GPM}: $\mG$,
      \texttt{query}: $Q := \set{(r_q, \bm{x_q}, z_q)}_{q=q_1}^{q_n}$,
      \texttt{evidence}: $E := \set{(r_e, \bm{x_e}, z_e)}_{e=e_1}^{e_m}$}
      \small
      \For{$B \in \pi$}
        \Comment{for each block $B$ in the variable partition}
        \Let{$l$}{$\textsc{Compute-Cluster-Probabilities}(B)$}
          \Comment{retrieve posterior probabilities of proposed clusters}
        \Let{$K$}{$|l|$} 
          \Comment{compute number of proposed clusters}
        \For{$q \in (Q \cap B)$}
          \Comment{for each query variable in block $B$}
            \If{$\texttt{data-type}(q) = \text{binary}$}
              \Comment{for binary variables}
              \Let{$p_G(x_{(r,q)}| \phi_{(z_r=k,q)})$}{$
                \textsc{Bernoulli}(x_{(r,q)};
                \frac{\beta_1 + |\set{x_{(s, q)}: x_{(s, q)}=1, z_s = k}|}
                {\beta_0 + \beta_1 + |\set{x_{(s, q)}: z_s = k}|}) $
              } \Comment{\hl{Is this inline math too detailed?}}
            \EndIf
        \EndFor
      \EndFor
    \EndFunction

  \end{algorithmic}
\end{algorithm}


\section{PseudoCode for CrossCat Incorporate} 
-- Inline math for incorporate
\section{PseudoCode for CrossCat Logpdf-Set}
-- Make context as an argument
\section{PseudoCode for Joint Logpdf-Set}
-- Remove helper
\section{PseudoCode for Relevance Search}
-- Add missing Unincorporate 


\end{document}



%%% Local Variables:
%%% mode: latex
%%% TeX-master: t
%%% End:
